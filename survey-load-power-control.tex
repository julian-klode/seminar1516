
\subsection{Combining Power control and Load balancing}

\subsubsection{Dynamic Source Routing Power-Aware (DSRPA)}
\label{dsrpa}
Dynamic Source Routing Power-Aware (DSRPA)\cite{djenouri2006new} defines a
power-aware variant of dynamic source routing.
 It trades of battery freshness and power consumption by defining a
metric on which it bases its routing.

Power consumption is reduced by recording the transmission power in route
request messages and then using that value and the reception power to calculate
the required power for transmitting a packet to the sender at the receiver side.

A node can share its power state by broadcasting in its neighbourhood a special
$E_{state}$ packet that contains the current power value. This packet is send
each time a certain percentage of power has been consumed. For example, if the
rate is set to $25\%$, a packet will be sent after 25, 50, and 75 per cent of
power consumed.

The metric representing the optimality of a a route $C=I_{0}, \ldots, I_{n-1}$
is defined as the minimum of $Opt_{c}$ (\ref{eq:dsrpa:opt-c}), where
\begin{itemize}
    \item $\alpha$ is a weight
    \item $eng(I_{i}) \in [0,1]$ is the rate of energy consumed by $I_{i}$,
    \item $Pow(I_{i}, I_{i+1})$ is the power required for the link from $I_{i}$ to $I_{i+1}$, and
    \item $MaxPow(I_{i})$ is the maximum power of the node $I_{i}$, that is, the power that allows
          it to cover its power range.
\end{itemize}

\begin{align}
Opt_{c} &:= \sum_{i=0}^{n-2} \frac{\alpha}{\underbrace{1-eng(I_{i})}_{\text{residual energy rate}}} + \frac{1-\alpha}{\underbrace{1-\frac{Pow(I_{i}, I_{i+1})}{MaxPow(I_{i})}}_{= 1 - \text{rate of total power}}}
\label{eq:dsrpa:opt-c}
\end{align} 


The more $\alpha$ increases, the more nodes with fresh batteries are preferred
(and the more $\alpha$ decreases, the more min-power routes are preferred).

$\alpha$ can be defined as a function of time (\ref{eq:dsrpa:alpha}),
where $EngDiff \in [0,1]$ is the difference between the maximum and minimum rate
of energy consumed in the network and $\alpha_{0}$ some initial value.
\begin{align}
    \alpha &:= \sqrt{(1-\alpha_{0})^{2} \cdot EngDiff + \alpha_{0}}
    \label{eq:dsrpa:alpha}
\end{align}

This counters the increasing difference $EngDiff$ because if that
difference increases, $\alpha$ increases, and thus fresher nodes are preferred,
and therefore, the value of $EngDiff$ decreases.

DSRPA disperses data over multiple routes.  While normal DSR only uses one
`optimal' route, DSRPA can discover and use multiple nodes if the route request
packet contains a \texttt{ALLROUTES} flag.

It will then use non-optimal routes to distribute the load of the network like
this:

If $\#packets$ is the number of packets to be sent and
$\#paths$ is the number of paths from the source to the
destination, and the routes are ordered according to their optimality,
with 0 being the best, then the number of packets to be sent on the route $i$
is given by (\ref{eq:dsrpa:packets}).
\begin{align}
     \#packets_{i} &:= \left\lceil \frac{2^{\# paths - i - 1} \cdot \# packets}{2^{\#paths} - 1} \right\rceil
     \label{eq:dsrpa:packets}
\end{align}

This ensures that most packets are delivered on the optimal route $i=0$, and
the package count in a less optimal route is less than the count in a more
optimal route. The number of packets send on a route $i$ is approximately
twice as much as the number of packets send on the route $i+1$.

The data dispersal requires more energy than would be needed without it. It
does however minimise the difference in energy states. The decision of whether
to use data dispersal or not must be made on a per-network basis: Networks
enjoying high connectivity will consume more power with data dispersal than
networks with low connectivity, because there will be a lot more broadcasting.
