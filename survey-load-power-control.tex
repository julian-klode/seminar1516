\subsection{Combining Power control and Load balancing}
\subsubsection{Dynamic Source Routing Power-Aware (DSRPA)}
DSRPA\cite{djenouri2006new} defines a power-aware variation of dynamic source
routing. It trades of battery freshness and power consumption by defining a
metric on which it bases its routing.

\todo[inline]{Define metric}

It also uses disperses data over multiple routes.
While normal DSR only uses the optimal route,
DSRPA will also use non-optimal routes to distribute the load of the network. It
does this similar to a process scheduler in UNIX.

If $\#packets$ is the number of packets to be sent and
$\#paths$ is the number of paths from the source to the
destination, and the routes are ordered according to their optimality,
with 0 being the best, then
\[ \#packets_{i} := \left\lceil \frac{2^{\# paths - i - 1} \cdot \# packets}{2^{\#paths} - 1} \right\rceil \]
is the number of packets to be sent on the route $i$.

This ensures that most packets are delivered on the optimal route $i=0$, and
the package count in a less optimal route is less than the count in a more
optimal route. The number of packets send on a route $i$ is approximately
twice as much as the number of packets send on the route $i+1$.
