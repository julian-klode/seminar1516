\subsection{Load distribution}\label{load-distribution}
Power-aware load distribution (or `control') algorithms distribute the
traffic based on the energy levels of the routers.
They do not try as much as the other algorithms to minimise the power required,
but instead focus more on distributing the power requirements among the available (routing) nodes.

\subsubsection{Local Energy Aware Routing Protocol (LEAR)}
LEAR\cite{woo2001non} is a variant of DSR.
Its main behavioural change lies in the forwarding of \textit{route-request} messages.
In LEAR, nodes only forward \textit{route-request} messages if their residual
energy level is above a certain threshold. Otherwise, the message is dropped
and the node does not take part in forwarding packages.

The threshold value is defined locally and changes with the time, decreasing
as the battery level decreases. If a source does not get a reply to its
route-request message, it will resend it. Nodes receiving the message multiple
times then lower their threshold value to allow the forwarding to continue.

DSR also employs a route cache. Obviously, just returning an existing node
from the route cache would not take energy level changes into consideration.
Thus, when a route is found in the cache, a new message called \textit{route-cache}
is sent via unicast to the next hop, which then forwards it to the end. This
produces far less traffic than the \textit{route-request} messages, which are
sent using broadcast messages.

\subsubsection{Conditional max-min battery capacity routing (CMMBCR)}
Like LEAR, CMMBCR\cite{toh2001maximum} uses a threshold level to optimise
the energy effiency of its routing.
Unlike LEAR, the threshold is fixed, though.

If all nodes in some paths have higher residual energy than the threshold,
the min-power path is chosen.
If all routes have nodes with a residual energy lower than the threshold, the
max-min route will be selected.

\subsubsection{Infra-Structure AODV for Infrastructured Ad-Hoc networks (ISAIAH)}
ISAIAH\cite{lindgren2002infrastructured} is an ad-hoc distance vector routing
algorithm. It routes similar to AODV, but selects routes that pass through
so called power base station (PBS) instead of mobile nodes, in order to reduce
the power consumption of the mobile nodes.

As with PARO, the routes chosen by this algorithm are longer than strictly necessary, but
because more nodes are stationary, the power consumption on the mobile nodes
is reduced.

\subsubsection{Energy Efficient Routing Protocol in MANET (EERP)}
EERP\cite{main2} is a variation of DSR that stores a minimum energy field
in a route request package, which is set to 100 percent at the beginning. Each
intermediate node will then override the value if its own residual energy rate
is lower than the stored one. The packet is then forwarded if the nodes residual
energy is above a certain threshold.

Once the route requests reach the destinations, the destination picks the route
with the maximum minimum hop energy, that is the \textit{max-min} path.
OMM\cite{li2001online} does something similar (see section~\ref{omm}), but EERP seems to use the power
levels before the transmission whereas OMM uses the power levels after the
transmission and can also switch to min-power paths.
