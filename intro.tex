Mobile ad-hoc networks present a number of unique challenges for routing protocols.
Their nodes are battery-powered and non-stationary, requiring flexible network architecture.
One example for ad-hoc networks, albeit not particularly mobile, are wireless sensor networks.

Routing protocols in such networks need to consider several important key
factors, such as:
\begin{enumerate}
   \item Scalability -- A protocol should scale to a large number of nodes
   \item Fault tolerance -- A protocol should continue working if some nodes fail
   \item Energy efficiency -- A protocol should not consume more energy than needed,
   either on an individual node, or aggregated.
\end{enumerate}

A recent survey\cite{alotaibi2012survey} on routing protocols for wireless
ad hoc networks describes six categories of routing protocols for wireless
ad-hoc networks,  in addition to the classic categories of proactive (that is, table driven)
and reactive (that is, on demand) routing protocols:
  Geographical, Geo-cast, Multi-Path, Hierarchical, Power-aware, Flow-Oriented,
  Hybrid, WMN, and Multicast.

That survey does not focus on energy-efficient networks, but intends to give
a broader overview. It unfortunately only discusses energy efficiency in the
context of power aware routing, although there a various routing protocols
focusing on energy-efficiency that are also part of the other categories.
For example, in section~\ref{gaf}, a routing protocol is discussed which
uses geographic features to reduce power consumption.
