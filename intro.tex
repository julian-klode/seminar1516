Initial work on ad-hoc networks focused on mobile sensor networks. With the
mass arrival of mobile wireless devices like smartphones, mobile ad hoc networks
are more important than ever, and the small batteries often shipped in those devices
require an energy efficient network, and thus, energy-efficient routing
protocols.

Apart from energy efficiency, routing algorithms should also consider other
factors, such as:
\begin{enumerate}
   \item Scalability --- A protocol should scale to a large number of nodes
   \item Fault tolerance --- A protocol should continue working if some nodes fail
\end{enumerate}

Section~\ref{categories} explains various categories of routing protocols for
wireless ad hoc networks, as identified by Alotaibi and Mukherjee\cite{alotaibi2012survey}. These
routing categories include the classic reactive and proactive routing categories,
as well as
  Geographical, Geo-cast, Multi-Path, Hierarchical, Power-aware, Flow-Oriented,
  Hybrid, WMN, and Multicast routing algorithms. 


Alotaibi and Mukherjee\cite{alotaibi2012survey} do not focus on energy-efficient networks, but intend to give
a broader overview. Thus, they unfortunately only discuss energy efficiency in the
context of power aware routing, although there are various routing protocols
focusing on energy-efficiency that are also part of (or related) to the other categories.
For example, in section~\ref{gaf}, a routing protocol is discussed which
uses geographic features to reduce power consumption.

Section~\ref{survey} explains the various categories of energy-efficient
routing in further detail, and gives a survey of the energy efficient protocols
listed in Figure~\ref{fig:overview}.

