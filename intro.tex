Mobile ad-hoc networks present a number of unique challenges for routing protocols.
Their nodes are battery-powered and non-stationary, requiring flexible network architecture.
One example for ad-hoc networks, albeit not particularly mobile, are wireless sensor networks.

Routing protocols in such networks need to consider several important key
factors:
\begin{enumerate}
   \item Scalability -- A protocol should scale to a large number of nodes
   \item Fault tolerance -- A protocol should continue working if some nodes fail
   \item Energy efficiency -- A protocol should not consume more energy than needed,
   either on an individual node, or aggregated.
\end{enumerate}

A recent survey\cite{alotaibi2012survey} on routing protocols for wireless
ad hoc networks describes six categories of routing protocols for wireless
ad-hoc networks,  in addition to the classic categories of proactive (that is, table driven)
and reactive (that is, on demand) routing protocols:
\begin{enumerate}
  \item Geographical
  \item Geo-cast
  \item Multi-Path
  \item Hierarchical
  \item Power-aware
  \item Flow-Oriented
  \item Hybrid
  \item WMN
  \item Multicast
\end{enumerate}

Most of these categories are not focused on energy efficiency, but more on
other factors such as fault tolerance, scalability and performance. The only
category in which energy efficiency is discussed is that of power-aware algorithms.
