\subsection{Transmission power control}
\label{power-control}
Protocols in this category try to minimise the power required to transmit
a message.

This class of algorithm is akin to a graph optimisation problem,
with the nodes in the MANET represented as nodes in a graph, the power required for
transmitting over a link $p_{ij}$ represented as the cost of a vertex, and the goal
to search the minimum-cost path between two nodes.

\subsubsection{Flow augmentation routing (FAR)}
Flow augmentation routing (FAR)\cite{chang2000energy} finds the routing
path in a (static) network that minimises the sum of link costs in the network,
where the link cost between $i$ and $j$ is defined as in (\ref{eq:far:cost}).
\begin{align}\label{eq:far:cost}
  C_{ij} = e_{ij}^{x_{i}}E_{i}^{x_{2}}R_{i}^{-x_{3}}
\end{align}
In (\ref{eq:far:cost}), \(e_{ij}\) is the energy cost for the transmission, $E_{i}$ is the initial energy
of the node $i$ and $R_{i}$ is the residue energy of the node. \(x_{1}, x_{2}, x_{3}\)
are weighting factors. For example, if $x_{1}=x_{2}=x_{3}=0$, then the cost
of each link is 1, and the optimal path is the shortest distance path.

In a static wireless network, $E_{i}$ and $e_{ij}$ are constant, but $R_{i}$
continues to decrease. Thus, the optimal link at one point in time might not be the
optimal link at a different point in time. FAR solves this problem iteratively:
Calculating the optimal path at each step, after updating the residual energy
and link costs.


\subsubsection{Power aware localized routing (PLR)}
%PLR-FIGURE
PLR\cite{stojmenovic2001power} is a geographical routing protocol, at least
to a certain extend, that tries to minimise the overall power usage. Each node
selects the next hop based on the power required for sending to this node, and
sending from that node indirectly to the destination.

For this, it has to know the locations of the next hops and the location of
the destinations. Based on the locations and the distance between the locations,
it can then estimate the power usage.

For direct connections, the power usage can be estimated as in (\ref{eq:plr:p}).
\begin{align}\label{eq:plr:p} p(d) := ad^{\alpha} + c \end{align}
In (\ref{eq:plr:p}), $a$ and $c$ are some constants and $\alpha \ge 2$ accounts
for the growth of power consumption in relation to the distance (power consumption
grows at least quadratic in relation to distance).

For indirect communication between the next hop and the final destination,
the power usage can be calculated as in (\ref{eq:plr:q}), where $n-1$ as in (\ref{eq:plr:n}) is the optimal number of intermediate nodes\cite{stojmenovic2001power}.
\begin{align}
  \label{eq:plr:q}
   q(d) &:= cn + da \left(\sfrac{a(\alpha - 1)}{c}\right)^{\sfrac{(1-\alpha)}{\alpha}}, \\
   \label{eq:plr:n}
      n &:= d\left(\sfrac{(a(\alpha - 1))}{c}\right)^{\sfrac{1}{\alpha}};
\end{align}

So, given an example like Figure~\ref{plrexample}, the intermediate node A
forwarding a package from $S$ to $D$ has to choose the next hop $N_{i}$ as the one for which
for which $p(\abs{AN_{i}}) + q(\abs{N_{i}D})$ is minimal.

%OOP
\begin{figure*}
\centering
\includegraphics[width=0.7\textwidth]{images/compow-level-choice}
\caption{Power choice in COMPOW: (a) too low, (b) too high, (c) optimal}
\label{compow:power-choice}
\end{figure*}

\subsubsection{Minimum energy routing (MER)}
MER\cite{doshi2002demand} modifies the dynamic-source-routing (DSR)\cite{johnson1996dynamic}
and 802.11 MAC algorithms\cite{woesner1998power} with 8 options (table~\ref{tbl:mer-options}) in order to
(i) obtain power information, (ii) measure overhead of energy-awareness, and
(iii) maintain the minimum power route with mobile nodes.

\begin{table}[b]
  \begin{tabular}{ll}
    Options & Implementation Level  \\
    \hline
    A: Routing packet-based power control & Routing software/\\ &802.11 firmware \\
    B: Minimum energy routing & Routing software \\
    C: Cache replies off & Routing software \\
    D: Internal cache options & Routing software \\
    E: Multi-hop route discovery & Routing software \\
    F: MAC layer ACK power control & 802.11 firmware \\
    G: Route Maintenance with power sensing & Routing software \\
    H: HMAC-Level snooping and gratious replies & 802.11 firmware \\
  \end{tabular}
  \caption{Options in MER}
  \label{tbl:mer-options}
\end{table}

Option $A$ modifies \textit{route-request} packages of the DSR protocol to
include the power used by the sender, which can the be added to the power
of the receiver to calculate the minimum power required for transmissions
from that sender to it. The value is appended at each intermediate node, and the
destination node then includes it in its \textit{route-reply} header.

The source node then includes this information when it sends a packet, and
all nodes transmit that package with the controlled power level.

Option F does the same for the ACK packets of the MAC layer.

Option $B$ is related to the route cache of the DSR protocol ($C$ and $D$ as well). It
uses the information obtained by option $A$ to select the route that requires
the minimum energy amongst its cached routes.

Option $G$ takes care of adjusting routes when the power transmission requirements
change because nodes moved.

Option $E$ and $H$ allow nodes not participating in the routing to recommend
alternative, more energy-efficient, routes.
\subsubsection{Local Minimum Energy Dynamic Source Routing Protocol (MEDSR)}
MEDSR\cite{tanque2007minimum} modifies the messages of the DSR protocol to
provide zero-overhead (as in: no extra control messages) energy-aware routing.

MEDSR uses two power levels for route discovery: A low one, and a high one. If
the source does not receive a reply for three route request messages at low
power, it switches to the high power level.

In contrast to other algorithms like MER, it only stores one power level field
in the packets, rather than one power-level per link. That power level is stored
in amended route request and route reply packets.

When a node $C$ receives a route reply from a node $D$, it can estimate the
power level required for transmitting to $D$ by calculating the minimum power
level as in (\ref{eq:medsr:pmin}).
\begin{align}
  \label{eq:medsr:pmin}
   P_{min} &:= P_{tx} - P_{recv} + P_{th} 
\end{align}
In order to avoid instability of the link, a margin is added,
leading to a modified definition (\ref{eq:medsr:pmin2}) of $P_{min}$.
\begin{align}
  \label{eq:medsr:pmin2}
   P_{min} &:= P_{tx} - P_{recv} + P_{th} + P_{margin}
\end{align}
In both of these equations (\ref{eq:medsr:pmin}, \ref{eq:medsr:pmin2}), $P_{tx}$ is the transmit power of $D$, $P_{recv}$ the receive power of $C$,
and $P_{th}$ is the threshold power for a successful receive. In 802.11, that is
usually $3.652 \cdot 10^{-10}$ watt.

In addition to the usual tables maintained by DSR, MEDSR also stores a power
table at each node: That is, for each node, it stores the minimum power required to reach it.



\subsubsection{Retransmission-energy aware routing (RAR)}
Transmissions may have link errors which require re-transmitting packets over
a link. A path with many short links might be more efficient than a path with
few long distance links in the absence of link errors; but with link errors,
things might look different, because some packets need to be re-transmitted.

RAR\cite{banerjee2002minimum} introduces a modified cost/power requirement
$P_{i,i+1}$ (\ref{eq:rar:linkcost}) of a link $p_{i,i+1}$ by including the error
rate of that link $e_{i,i+1}$ .
\begin{align}\label{eq:rar:linkcost}
  P_{i,i+1} = \underbrace{p_{i,i+1}}_{\text{original link cost}} \cdot \underbrace{\sfrac{1}{\overbrace{(1-e_{i,i+1})}^{\text{success rate}}}}_{\text{expected number of tries}}
\end{align} 

\subsubsection{Smallest common power (COMPOW)}
Network protocols often require handshaking of some sorts, like acknowledgements
sent to a sender for his packages. Thus, for a MANET to be reliable for those
protocols, for every two nodes, there must be routes in both directions, that
is, some form of bi-directionality.

The smallest common power protocol\cite{narayanaswamy2002power} achieves bi-directionality
of communication in the network.
It does so by maintaining a smallest common transmission power for all nodes,
chosen so that all nodes can communicate with each other. If the power is too
low, some nodes could not reach some others, if it is too high, there might be
many redundant paths. See Figure~\ref{compow:power-choice}.

COMPOW assumes a discrete set of power levels $P_{i}$ and maintains routing
tables ${RT}_{P_{i}}$ containing all nodes reachable from the current one. The
optimal power level $P_{i}$ can be determined as the smallest one for which
$\abs{RT_{P_{i}}}$ is equal to the number of nodes in the network, by
exchanging the routing tables between nodes.


\subsubsection{PARO}
The PARO Protocol\cite{gomez2003paro} introduces intermediate nodes into a
route, increasing the length of the route, in order to reduce the transmission
power required by each node.

In the PARO model, all successfully received packets are acknowledged on the
link layer, and nodes must be able to overhear transmissions of other nodes
above a certain minimum signal-to-noise-ratio (SNR) value. They must also
be able to determine the SNR value of a received packet. It is also assumed
that the power required for a transmission from $A$ to $B$ is about the same
as from $B$ to $A$, as such interference conditions must be similar in both
directions, requiring the use of appropriate interference-free media access
control (MAC) such as found in 802.11.

In PARO, a node keeps it transmitter on for sending a packet for $L/C$ seconds,
where $L$ is the number of bits to be transmitted and $C$ is the speed of the
channel in bits per second.
It also keeps it transmitter on for sending acknowledgements of size $l$.

Given a route $k$, the power needed to transmit a packet over that
route can thus be defined as in (\ref{eq:paro:P}).
\begin{align}\label{eq:paro:P}
  P_{k} := \sum_{i=0}^{N_{k}} \frac{T_{i,i+1} \cdot L + T_{i+1,i} l}{C}
\end{align}
The factor $T_{i,j}$ in (\ref{eq:paro:P}) defines the minimum transmission power
to be used at $i$ so that the packet can still be received by $j$, and $N_{k}$
is the number of times the packet is forwarded along the route.

This equation only considers transmission power and not the power required for
listening or processing data, no attempt is made to minimise those.

Let $R_{k}$ be the power required for discovering a route $k$ for which $P_{k}$
is the minimum (the `best' route). Then, transmitting $Q$ packets along the best route $k$ can be
defined as in (\ref{eq:paro:c}).
\begin{align}\label{eq:paro:c}
  C_{k} := R_{k} + Q P_{k}
\end{align}


Packets in PARO store the transmission power and other nodes overhearing the
packets use that information together with the received power to estimate the
power required to transmit to that node.

PARO discovers routes on-demand instead of creating them in advance. This means
that the first packet from a source to a destination will be sent directly. Nodes
overhearing the transmission can elect to become a redirector, by sending a
\textit{route-redirect} message to the source and destination nodes. This can
happen with every further packet, and thus new redirectors can be added with
each packet, eventually converging towards a final route achieving the minimal
transmission power as defined in (\ref{eq:paro:P}).
