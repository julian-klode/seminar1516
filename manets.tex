
\documentclass[conference]{IEEEtran}
\usepackage[utf8]{inputenc}
\usepackage[T1]{fontenc}
\usepackage[UKenglish]{babel}
\usepackage{cite}
\usepackage{graphicx}
\usepackage[cmex10]{amsmath}
\usepackage{hyperref}
\parskip 2ex


\begin{document}
\title{Energy-efficient Routing Protocols for Mobile ad-hoc Networks}
\author{\IEEEauthorblockN{Julian Andres Klode}
\IEEEauthorblockA{Philips-Universität Marburg \\
Email:\href{mailto:klode@mathematik.uni-marburg.de}{klode@mathematik.uni-marburg.de}}}

\maketitle


\begin{abstract}
Mobile Ad-Hoc networks are blah blah.
\end{abstract}


\section{Introduction}
Mobile ad-hoc networks present a number of unique challenges for routing protocols.
Their nodes are battery-powered and non-stationary, requiring flexible network architecture.
One example for ad-hoc networks, albeit not particularly mobile, are wireless sensor networks.

Routing protocols in such networks need to consider several important key
factors:
\begin{enumerate}
   \item Scalability -- A protocol should scale to a large number of nodes
   \item Fault tolerance -- A protocol should continue working if some nodes fail
   \item Energy efficiency -- A protocol should not consume more energy than needed,
   either on an individual node, or aggregated.
\end{enumerate}

A recent survey\cite{alotaibi2012survey} on routing protocols for wireless
ad hoc networks describes six categories of routing protocols for wireless
ad-hoc networks,  in addition to the classic categories of proactive (that is, table driven)
and reactive (that is, on demand) routing protocols:
\begin{enumerate}
  \item Geographical
  \item Geo-cast
  \item Multi-Path
  \item Hierarchical
  \item Power-aware
  \item Flow-Oriented
  \item Hybrid
  \item WMN
  \item Multicast
\end{enumerate}

Most of these categories are not focused on energy efficiency, but more on
other factors such as fault tolerance, scalability and performance. The only
category in which energy efficiency is discussed is that of power-aware algorithms.


\section{Categories}
Classic routing protocols are categorised in two dimensions:
centralized/distributable and proactive/reactive.
The unique challenges inherent in wireless ad-hoc networks led to a larger
amount of categories, which are not always necessarily disjoint, meaning
that an algorithm may belong to one or more categories.

\subsection{Basics: Proactive routing}
In proactive routing algorithms, each router builds its routing table by
(regularly) exchanging messages with other routers in the network.

This has the advantage that routing information is already available when a
package is to be routed.
A huge disadvantage of proactive routing protocols is the overhead imposed
by the exchange of the update messages.

\subsection{Basics: Reactive routing}
In a reactive routing algorithm, routes are discovered when a packet arrives
from a source and needs to be delivered to some destination.
While the routes need to be discovered more often than in proactive algorithms,
there is no traffic overhead due to the lack of special control messages, and
as such, the reactive approach is considered to scale better.

\subsection{Geographical routing}
The basic idea behind geographical routing is that nodes are addressed by
their geographic location instead of an IP address. This has the advantage
that each node does not need to know the full network topology, but each
node needs to know its location and each source needs to know the location
of the receiver.

Geographical routing does not consider energy efficiency, as such it makes
no sense to discuss them further.

\subsection{Geo-cast}
Geo-cast routing merges multi-cast routing with geographical routing, to
deliver messages to a group of nodes identified by their locations.

\subsection{Hierarchical RA}
A hierarchical routing algorithm defines multiple zones or clusters with
gateways that connect them with each other. Inside a cluster, there often
are one or more cluster heads, that are responsible for maintaining the
connectivity within the cluster. Non-head nodes can only communicate with
their cluster heads, whereas gateway nodes can exchange information with
other clusters.


\subsection{Multi-path}
In a multi-path routing algorithms, multiple paths exists from one source
to a destination. The obvious advantages of a multi-path routing
algorithm include fault tolerance (if one path fails, another is still in
use), load distribution (congested routes can be avoided by chosing alternative
routes), peak performance (multiple routes may be used in parallel for different
parts of data).

Energy-efficiency is not a paramount concern of most multi-path routing, the
load balancing idea can however not only be applied to performance, but also
to power levels, to distribute the power used within the network, as we will
see later on. Such an algorithm would of course also be power-aware.

\subsection{Power-aware}
A power-aware routing algorithm tries to find a good trade off between
power consumption of nodes and mobility. \cite{main1} describes two categories
of power-aware protocols: those that try to minimise the active communication energy -- either
by controlling the power used for transmission, or by load control -- and
algorithms that try to minimise the inactivity energy, for example, by putting
rarely used nodes to sleep.

\subsection{Hybrid}
A hybrid protocol starts off as a proactive routing protocol but switches
to reactive routing for newly added nodes in order to reduce the control
overhead of proactive routing. In ad-hoc networks, it is implemented in
hierarchical network architectures.

\subsection{Flow-oriented}
\subsection{WMN}
\subsection{Multicast}

\section{Protocol survey}
\subsection{Transmission power control}
\subsubsection{Flow argumentation routing (FAR)}
Flow argumentation routing (FAR) was discussed by Chang and Tassiulas\cite{chang2000energy};
\subsubsection{Online max-min (OMM)}
OMM was discussed by Li, Alsam, Rus\cite{li2001online};
\subsubsection{Power aware localized routing (PLR)}
Discussed by Stojmenovic and Lin \cite{stojmenovic2001power};
\subsubsection{Minimum energy routing (MER)}
Discussed by Doshi, Bhandare, Brown \cite{doshi2002demand};
\subsubsection{Retransmission-energy aware routing (RAR)}
Discussed by Banarjee and Misra \cite{banerjee2002minimum};
\subsubsection{Smallest common power (COMPOW)}
Discussed by Narayanaswamy, Kawadia, Sreenivas and Kumar\cite{narayanaswamy2002power};

\subsubsection{PARO}
The PARO Protocol\cite{gomez2003paro} introduces intermediate nodes into a
route, increasing the length of the route, in order to reduce the transmission
power required by each node.

\subsection{Load distribution}
\subsubsection{Local Energy Aware Routing Protocol (LEAR)}
Woo et al. introduced LEAR\cite{woo2001non},

\subsubsection{Conditional max-min battery capacity routing (CMMBCR)}
Too introduced CMMBCR\cite{toh2001maximum}.

\subsubsection{Dynamic Source Routing Power-Aware (DSRPA)}
In DSRPA\cite{djenouri2006dynamic}, nodes with the freshest battery are
chosen to route the packages, in order to achieve connectivity for as
long as possible.\cite{alotaibi2012survey}

\subsection{Sleep/power­ down mode}
\subsubsection{SPAN}
Span\cite{chen2002span} is \ldots

\subsubsection{Geographic adaptive fidelity (GAF)}
GAF\cite{xu2001geography} is \ldots{}

\subsubsection{Prototype embedded network (PEN)}
PEN\cite{girling2000design} is \ldots{}

\subsubsection{Power-Aware Multi-Access Protocol with Signaling Ad-Hoc Networks (PAMAS)}
PAMA\cite{singh1998pamas} controls the power usage based on the activity of
a node: Nodes that are not involved in the transmission of packages are turned
off for some time.

\subsection{Uncategorised}

\subsubsection{Energy Saving Dynamic Resource Routing Protocol (ESDSR)}
ESDSR\cite{tarique2005energy} \ldots
\subsubsection{Local Minimum Energy Dynamic Source Routing Protocol (MEDSR)}
MEDSR\cite{tanque2007minimum} \ldots

\subsubsection{Energy Efficient Routing Protocol in MANET (EERP)}
Suvarna and Naik introduced EERP, a routing protocol with a horribly
generic name\cite{main2}.

\section{Conclusion}
There are many different approaches to energy-efficient routing in
mobile ad-hoc networks. Current research seems to focus on protocols
that minimise active communication energy, and mostly on those that
rely on power control rather than load control.


\bibliographystyle{IEEEtran}
\bibliography{IEEEabrv,manets}

\end{document}


