There are many different approaches to energy-efficient routing in
mobile ad-hoc networks. Current research seems to focus on protocols
that minimise active communication energy, and mostly on those that
rely on power control rather than load control.

We have seen various ways to minimise the energy required for sending
messages, for example, PARO, which introduces closer intermediate nodes,
in order to reduce the distance between nodes, and thus their
power consumption.

We have taken a look at various load distribution algorithms that distribute
the load according to the energy levels of the routers. Especially if not all
nodes in an ad-hoc network are mobile, this approach makes sense: It is
pointless to deliver messages through battery-driven routers when they can
also be forwarded by stationary nodes that are connected to a power supply.

We have seen two basic approaches to sleep/power-down optimisation, master-slave
and the PEN protocol. Out of these, the master-slave approach seems like the
best choice for most networks, due to the almost-always available master nodes,
whereas the PEN protocol is only useful for networks without much interaction,
due to senders having to wait for destinations to be up.
